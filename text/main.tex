\documentclass[11pt]{article}

\usepackage{graphicx}

% Margins
\topmargin=-0.45in
\evensidemargin=0in
\oddsidemargin=0in
\textwidth=6.5in
\textheight=9.0in
\headsep=0.25in


\title{ Robert Soul\'{e} --- CCR Editor Vision}
\author{  }
\date{ }

\begin{document}

\maketitle	
\thispagestyle{empty}

SIGCOMM CCR holds a unique place among ACM and SIGCOMM publication venues due to its focus on three classes of contributions. First, as a type of newsletter, CCR is a venue for disseminating information and a forum for prompting discussion about the community, i.e., editorial notes. Second, CCR is a forum for promoting computer networking and computer systems education. Finally, CCR is a selective venue to publish technical results. Each of these aspects are an important part of the CCR agenda. I am unaware of any alternative venues for the first two classes of contributions. However, there are many technical venues where authors can publish their work, and so those submissions merit a more in-depth discussion.  

Within the networking research community, there is currently a strong focus on practicality and operational experience, as opposed to pushing bold new ideas that may not be readily deployable or fully validated. If we are to grow as a research community and continue to push boundaries, we need to explore these ideas that may be out of the mainstream. My vision for CCR is that it can promote contributions that are more thought-provoking, provocative, or contentious than what would appear at a typical conference or workshop. 

My belief in the need for provocative ideas is not just abstract, but personal. My pre-tenure career was largely spent on programmable networks and what is now called "in-network computing". The first paper I published on the topic was in CCR, "Paxos Made Switch-y", April 2016, in which we described how one could implement Paxos on P4-programmable switches. This idea would have been well-received at mainstream conferences, and, indeed, when I spoke with Nick McKeown about it later, he told me, "that idea really came out of left field". Since that publication, there have been a number of in-network computing papers published at the main conferences, but I attribute CCR as being the place where we could first get the idea out to the community. Part of why the CCR Editor position is attractive to me is because I credit CCR as being the “cradle” where many of the ideas that formed the foundation for my research career were born. Going forward, I hope that CCR can continue to play the role of a forum for discussing interesting/novel/provocative ideas that are slightly out of the mainstream. 

Of course, this raises the question of how to realize that vision. Partially, this can be accomplished by editorial selection, community reinforcement, and by continuing activities like the recognition of “Best of CCR” at SIGCOMM. One idea I had that would help establish CCR as a type of  “taste maker” for networked systems would be  to establish a new regular article/series to appear in CCR that I tentatively call, “What are you reading?”. For each CCR issue, I would ask a prominent networking researcher to identify a recent or historical paper that they found particularly interesting or thought provoking, and write a paragraph or two on what they found stimulating. I believe that highlighting such papers would inspire others in the community to pursue ambitious ideas.


\end{document}
