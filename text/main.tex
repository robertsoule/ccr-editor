\documentclass[11pt]{article}

\usepackage{graphicx}

% Margins
\topmargin=-0.45in
\evensidemargin=0in
\oddsidemargin=0in
\textwidth=6.5in
\textheight=9.0in
\headsep=0.25in


\title{ Robert Soul\'{e} --- CCR Editor Vision}
\author{  }
\date{ }

\begin{document}

\maketitle	
\thispagestyle{empty}


SIGCOMM CCR holds a unique place among ACM and SIGCOMM publication venues due to its focus on three classes of contributions. First, as a type of newsletter, CCR is a venue for disseminating information and a forum for prompting discussion about the community, i.e., editorial notes. A good example of this type of paper is “Rethinking SIGCOMM’s Conferences” Making Form Follow Function”, by Scott Schenker, October 2022. Second, CCR is a forum for promoting computer networking and computer systems education, publishing articles that have a type of tutorial aspect for situations in which there is a lack of alternative resources (e.g., textbooks, online references, etc.) available for interested parties. A good example of this type of contribution is “The I/O Driven Server: From SmartNICs to Data Movement Controllers” by Justine Sherry, October 2023. Finally, CCR is a selective venue to publish technical results, such as “An Analysis of QUIC Connection Migration in the Wild” by Aurelien and Cristel, April 2025.

In terms of technical contributions, there are many venues where authors can publish their work. My vision for CCR is that it can promote contributions that are more thought-provoking, provocative, or contentious than what would appear at a typical conference or workshop. I think that as a research community, there is currently a strong focus on practicality and operational experience, as opposed to pushing bold new ideas. I would like CCR to be a venue for promoting these bold ideas, even if they are not fully validated. I strongly believe that we need these types of “outside the mainstream” papers to grow the future of networked systems. 

To share a personal anecdote, my pre-tenure career was largely spent on programmable networks and what is now called ``in-network computing". The first paper I published on the topic was in CCR, "Paxos Made Switch-y", April 2016, in which we described how one could implement Paxos on P4-programmable switches. This was one of the first papers that explored in-network computing ideas and I don't think it would have been well-received at SIGCOMM/NSDI, etc.. Indeed, when I spoke with Nick McKeown about it later, he told me, "that idea really came out of left field". Since then, there have been a number of in-network computing papers published at the main conferences, but I attribute CCR as being the place where we could first get the idea out to the community. Part of why the CCR Editor position is attractive to me is because I credit CCR as being the “cradle” where many of the ideas that formed the foundation for my research career were born. Going forward, I hope that CCR can continue to play the role of a forum for discussing interesting/novel/provocative ideas that are slightly out of the mainstream. 

Viewed this way, I view CCR as a type of “taste maker” for networked systems. Towards this goal, I would like to establish a new regular article/series to appear in CCR that I tentatively call, “What are you reading?”. For each CCR issue, I would ask a prominent networking researcher to identify a recent or historical paper that they found particularly interesting or thought provoking, and write a paragraph or two on what they found stimulating. I believe that highlighting such papers would inspire others in the community to pursue ambitious ideas.






\end{document}
